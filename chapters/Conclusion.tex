\renewcommand\chaptername{}
\chapter{CONCLUSION}


The scope of this thesis was related to the usage of remote sensing and deep learning in forest inventory. The importance of forest taxation as well as the drawbacks of in-sute field assessments were discussed at the beginning of this work. A detailed literature review was done in order to estimate the current state of technology for trees detection and delineation. Automatic trees detection delineation pipeline, which was built during this project, demonstrates a promising perspectives for forest inventory. 
The hypothesis, stating that rescaled aerial imagery and satellite imagery will have similar domains for the neural network was confirmed, results show that state-of-the-art Convolutional Neural Network (\gls{CNN}) trained on rescaled aerial RGB imagery is able to delineate the tree crowns directly from the real satellite RGB imagery. \gls{ITC} delineation F1 score of 72\% and tree detection rate of 89\% confirm that it is possible to substitute the manual field assessments by the proposed technology in forest inventory. 486 hectare of experimental area (out of 1020 hectare) was used to generate dataset. Local maxima and Watershed algorithm altogether with different mathematical morphological operations were successfuly applied to obtain satellite spatial resolution (0.31m and 0.5m) imagery from the UAV-carried high-resolution aerial images. Fully-convolutional neural network UResnet34 model with pretrained values was trained on the training data, and hypothesis was confirmed by the test dataset and real satellite imagery. As it was discussed in Background literature section, there is a problem arising from the lack of datasets for various forest types and the performance of the model tested on different satellite images show that the model severely depends on the region, and it will change among different environments. However, the results of this research show that growing use of \gls{UAV}s and availability of high-resolution imagery combined with generalization ability of deep neural networks will eventually make it possible to obtain good quality forest inventory information from large scale satellite imagery.